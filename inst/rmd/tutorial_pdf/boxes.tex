% Using xcolors with dvipsnames before mdframed to avoid option clash
\usepackage{xcolor}
\usepackage[framemethod=TikZ]{mdframed}

% Defining the base style of the box as bbox
\mdfdefinestyle{bbox}{
  leftmargin = 0cm,%
  rightmargin = 0cm,%
  innerleftmargin = 0.5cm,%
  innerrightmargin = 0.5cm,%
  roundcorner = 5pt,%
  linewidth = 2pt,%
  nobreak=false%,
  %needspace=1in % Trying to avoid a page break after the title but does not do the trick...
}

% Defining custom box environment (cbox) with 2 arguments:
% The first optional (defaults to empty string) is the title of the box
% The second (mandatory) defines the color of the box (linecolor and background is linecolor!30)

% body: fill (#2), colour (#3)
% header: fill (#4), colour (#5)
\newenvironment{cbox}[5][]{
  \def\temp{#1}\ifx\temp\empty
    \begin{mdframed}[style=bbox,%
      linecolor = #4,%
      fontcolor = #3,%
      innertopmargin = 10pt,%
      skipabove = 10pt,%
      backgroundcolor = #2]%
  \else
    \begin{mdframed}[style=bbox,%
      linecolor = #4,%
      fontcolor = #3,%
      backgroundcolor = #2,%
      innertopmargin = 3pt,%
      skipabove = 10pt,%
      frametitle={\tikz[baseline = (current bounding box.east), outer sep = 0pt]
                  \node[anchor = east, rounded corners = 3pt, rectangle, color = #5, fill = #4]{#1};},%
                        frametitleaboveskip = \dimexpr-\ht\strutbox\relax]
  \fi
}{\end{mdframed}}

\mdfdefinestyle{input}{
  frametitle = {},%
  leftmargin = 0cm,%
  rightmargin = 0cm,%
  innerleftmargin = 0.2cm,%
  innerrightmargin = 0.2cm,%
  innertopmargin=0.5em,%
  roundcorner = 2pt,%
  linecolor = lightgray,%
  linewidth = 1pt,%
  backgroundcolor = lightgray!30
}

% Defining cbox macro to draw a coloured block
% around the content

\newcommand{\cboxs}[5][]{\begin{cbox}[#1]{#2}{#3}{#4}{#5}}
\newcommand{\cboxe}{\end{cbox}}

\makeatletter
